\def\year{2015}
%File: formatting-instruction.tex
\documentclass[letterpaper]{article}

% Required Packages
\usepackage{aaai}
\usepackage{times}
\usepackage{helvet}
\usepackage{courier}
\frenchspacing
\setlength{\pdfpagewidth}{8.5in}
\setlength{\pdfpageheight}{11in}

% Section numbers. 
\setcounter{secnumdepth}{2}  

\nocopyright
\begin{document}
% Title and author information
\title{How to Kill a Hamster - Group NULL\\ 02285 AI \& MAS}
\author{Daniel Schougaard \\ \textit{s103446} \And Kasper Reindahl Rasmussen\\ \textit{s103476} \And Martin V. Ottesen\\ \textit{s060186}}
\maketitle

\begin{abstract}
The abstract goes here. Please read this document carefully before preparing your manuscript.

To ensure that all reports have a uniform appearance corresponding to published papers at the major AI conferences (like IJCAI and AAAI), authors must adhere to the following instructions. 
\end{abstract}

\section{Introduction}
	Like so ~\cite{book2015}. 
\section{Background}
	Theory : Relaxation

\section{Related Work}

\section{Methods}
	\subsection{Brief Overview of Design and Implementation}
		Reindahl\\
		Randomization
	\subsection{Search}
		\subsubsection{The Heuristic}
		Reindahl \\
		RELAXATION OF A RABBIT



	\subsection{Conflict Resolution}
		Since we're using multiple agents, each planning their own route, conflicts are bound to happen. We've chosen to rely on a method to resolve these, somewhat similar to online planning. But first and foremost, we need to describe exactly how we \emph{detect} these conflicts.

		\subsubsection{Detecting the Conflicts}
			Currently we rely on two different methods for detecting conflicts. The first method is simply letting the server \emph{tell} the client, that there is a conflict. This is by far the simplest possible way to detect a conflict, but it does inevitably result in a grater overhead. 

			The second approach is, that we employ a sort of double search. First we relax the domain, removing all other agents and boxes, that the current agent can not move. This results in a new subdomain, in which a search is performed. This results in a sort of threshold: In a perfect world, a solution for this problem exists in $X$ moves. This is then again used, to cut off the normal search, once it has taken too long. When this happens, we assume that it is because of a conflict. Using the route of the relaxed search, we identify which objects are in the way, causing this conflict.

		\subsubsection{Types of Conflicts}

			Having detected a conflict, we partition the causes into one of two groups: The cases where it is only a box in the way, and the cases where it is a box. As a rule, all boxes found in the route, is handled first. This is done to avoid cases, where an agent otherwise asked to move a box, is told to get out of the way. An example of this, can be seen below. $0$ is a \verb=BLUE= agent, and $1$ is a \verb=RED= agent. The box $B$ is also \verb=RED=. 
			\begin{verbatim}
				...+++++++++...
				     0 B1
				...+++++++++...
			\end{verbatim}
			In this example, $0$ would like to move right. Obviously, that is impossible, due to the fact that both the agent $1$ and the box $B$ is in the way. Should we chose to handle agents first, $1$ would move out of the way, leaving no one to move the box $1$.

		\subsubsection{Box Resolution}
			First we identify which agent is supposed to move any given box.  

			Agent close by not doing anything






		\subsubsection{Cycle Detection}

		Daniel

\section{Results and Experiments}
	\subsection{Distance Maps}
	\subsection{Goals}
		\begin{itemize}
			\item{Individual goals vs series of goals}
			\item{Prioritizing goals}
		\end{itemize}

	\subsection{Comparison of Levels}
		\begin{itemize}
			\item{Why is this?}
			\item{Large levels!}
			\item{Coordination amongst Agents.}
		\end{itemize}

\section{Discussion}

\section{Future Work}
	\begin{itemize}
		\item{Large Levels}
		\item{Coordination}
	\end{itemize}

\section{References}
	Martin\\
	6 references, 3 papers.
		
		
% References and end of paper
\bibliographystyle{aaai}
\bibliography{bibliography}
\end{document}