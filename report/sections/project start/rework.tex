%Has this been done before? What is the closest related research? How does your work difer? Related work is sometimes integrated into the introduction or background, and sometimes it is made a separate section towards the end of the paper. To make the related work section, you will be required to do some literature search to see if you can find papers that use similar methods (or combinations of methods) on similar types of problems. A piece of related work could for instance be if someone wrote a paper on using similar methods for the Sokoban domain. Sometimes it can be hard to find related work, but you should do your best. Use e.g. Google and Google Scholar (if a certain relevant paper is licensed, try to download it via findit.dtu.dk).
\section{Related Work}
Sokoban is as stated earlier a classical automated planing problem it was created in 1981 by Hiroyuki Imabayashi, and published in 1982. During it's 33 years a myriad of solutions have been created to solve it.
The Pukoban and Multi-Agent variations however are not as well documented.
\subsection{Pukoban}
~\cite[Agent motion planning with pull and push moves]{Pukoban} is a solution to a single agent to the Pukoban problem. To the single agent implementation this solution has a very similar approach it however uses the Kuhn-Munkres(Hungarian)algorithm, instead of Floyd-Warshall this gives them some overhaed as they have to runn it multiple times.
~\cite{Pukoban} pairs up the boxes to goals they also analyse for bottlenecks called chock points and clean them for what they call Clogs, they have no partial order planing   so they only have a success rate of 30 on fairly simple levels.
\subsection{Multi-Agent}
~\cite[A Multi-Agent Planning Approach Integrated with Learning Mechanism]{Multi-Agent} is a solution to Sokoban problems it uses centralized planing. It is integrates machine learning in order to build up a knowledge base capable of solving complex Sokoban problems. As the problems the self learning system is solving is Sokoban the state-space does not explode in the same way as a Pukoban would, this makes it possible fairly fast to go throw possible combinations and learn what no to do, and be rewarded in accordance to how many goals were attained before the game was deadlocked. It is hard to compare this approach to the one we are using for the Pokuban problem but for this type of problem it should theoretically be more suited as it could with enough time solve a NP-hard problem, and the central control system is a better choice for Sokuban as backtracking is not a possibility.