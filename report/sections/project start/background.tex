%Briefly present the theories that you work relies on. Note that you are allowed to expect the reader to be familiar with everything presented in the course curriculum, but you should at least mention which of these theories you are using and make suitable references. Even for the theories in the course curriculum it might also be necessary to settle the notational conventions, as these sometimes differ between references. If you use theories and ideas outside the course curriculum, you should explain them in a bit more detail, and of course also make suitable references.
%Theory :	Automated planning
%			Multi-agent systems
%			Relaxation
systems.
\section{Background}
The Pukoban problem is closely related to the classical automated planing problem called Sokoban the difference is that you are also allowed to pull the boxes in the Pukoban case. As an automated planing problem we need to calculate action/command sequences to solve the problem/levels these sequences of commands have to be sent to to a provided server application.
The levels are solved using progression planning, implemented using a best-first heuristic search to explore the state space as the roles in the assignment implies completeness in the levels. The search algorithms we have found interesting to look at are the AStar and Greedy as they are well documented and suit our purpose well.
\subsection{Multi-agent System}
As part of the solution we have to take the requirement for multiple agents into consideration, this means that some of the basic assumptions in automatic planing have to be re-evaluated these are normally single-agent by default.
Multi-agent systems are generality neither static or fully observable this makes offline planing problematic as inter agent coordination or centralised planing is very heavy computationally, an alternative to this is to make the agents work autonomously, this however opens op the possibility of agents obstructing each-other, furthermore the introduction box restrictions on which agents are able to move them can make some goals unattainable, this opens up the requirement for some sort of conflict resolution in the form of online replanning.
\subsection{Relaxation}
In order to make sure that an agent has a plan to carry out, even though none of it's goals are actually attainable a relaxation of the problem may be necessary to relax the search space in some way. The relaxation has to be done in such a the plan is still feasible, so that the time spent on calculation a relaxed solution is not a wast of time and processing power only non-permanent obstacles should be disregarded and  agent and box parameters are final. By following these rules a we can end up with an emulated single agent system.
Relaxed solution should however only be used if no non relaxed solution can be produced. 
\subsection{Partial-order Planing}
An other useful relaxation is only to plan on a subset of the goals at a time instead of trying to solve them all in one go, at the same time if you only look at one goal at a time the agents might constantly undo attained goals in order to fulfil new ones. In order to make this function optimally ordering of goals is also necessary so that the goals come in an optimal order ~\cite{Subgoals}.
